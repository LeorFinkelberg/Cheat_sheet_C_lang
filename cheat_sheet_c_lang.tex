\documentclass[%
	11pt,
	a4paper,
	utf8,
	%twocolumn
		]{article}	

\usepackage{style_packages/podvoyskiy_article_extended}


\begin{document}
\title{Приемы программирования на языке C}

\author{}

\date{}
\maketitle

\thispagestyle{fancy}

\tableofcontents

\section{Ресуры по языку C}

\url{https://learnc.info/c/}

\section{Visual Studio Code как среда разработки для языка C}

Скачать Visual Studio Code можно здесь \url{https://code.visualstudio.com/}. Для ОС Windows нужно еще установить GCC. На ОС Linux компилятор gcc доступен <<из коробки>>. На ОС MacOS компилятор gcc можно установить с помщью утилиты \verb|brew|.

После установки IDE останется только создать директорию с проектом под язык Си. Когда Visual Studio Code увидит файл с расширением \verb|*.c|, она предложит установить специальное расширение <<C/C++ Extension Pack v1.X.X>>.



% Источники в "Газовой промышленности" нумеруются по мере упоминания 
\begin{thebibliography}{99}\addcontentsline{toc}{section}{Список литературы}
	\bibitem{koltzov-c-lang:2019}{ \emph{Кольцов Д.М.} Си на примерах. Практика, практика и только практика. -- СПб.: Наука и Техника, 2019. -- 288 с.}
\end{thebibliography}

%\listoffigures\addcontentsline{toc}{section}{Список иллюстраций}

\lstlistoflistings\addcontentsline{toc}{section}{Список листингов}

\end{document}
